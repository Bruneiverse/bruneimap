\documentclass[12pt]{article}
\usepackage[a4paper, margin = 1in]{geometry}
\usepackage{graphicx, amsmath, booktabs}
\usepackage{tcolorbox}
\usepackage{hyperref}
\usepackage{booktabs}
\usepackage[table,xcdraw]{xcolor}
\usepackage{biblatex} 


\title{Spatial data analysis of school (locations?) in Brunei Darussalam }
\author{Alvin Bong JL, Aniq Najwa, Danish Ikhwan, Rayme Hijazi}
\date{November 2024}

\setlength{\parskip}{1em} % Spacing per paragraph
\setlength{\parindent}{0pt} % Remove paragraph indentation globally
\begin{document}

\maketitle
\begin{abstract}
\noindent
    This study analyzes the spatial distribution of schools across Brunei Darussalam to assess how well current educational facilities serve both urban and rural populations. Using spatial autocorrelation techniques like Moran’s I and the Getis-Ord \( G_i^* \) statistic, this research identifies clusters of schools and evaluates their alignment with population density. Findings indicate significant school clustering in urban areas, particularly near Bandar Seri Begawan, with comparatively fewer facilities in rural regions. This imbalance suggests potential challenges in accessing education for students in less populated areas. The study offers valuable insights for policymakers to improve resource distribution, aiming to support equitable access to education and inform future spatial planning in educational infrastructure. 
\end{abstract}

%keywords section
%cover page for report

\section{Introduction}
\label{sec: intro}
Education plays a critical role in fostering social and economic development, and equitable access to educational facilities is essential for ensuring equal opportunities for all students. In Brunei Darussalam, education is free for citizens and compulsory from ages 5 to 16. The country’s high literacy rate, robust educational system, and strong economic standing position it as a leader in Southeast Asian education. However, while Brunei’s achievements in educational access are widely recognized, little research has been conducted to determine whether the spatial distribution of schools is optimized to serve both urban and rural populations effectively. 

This study seeks to explore the distribution of schools in Brunei through three main questions:
\begin{enumerate}
    \item Are there clusters of schools within certain areas of Brunei?
    \item Where are these clusters located?
    \item Do these clusters align with population density?
\end{enumerate}

Brunei’s administrative divisions include districts, mukims (sub-districts), and kampongs (villages), each with varied population densities and geographic characteristics. For example, population concentrations are generally higher along the coastlines, particularly in the capital, Bandar Seri Begawan, while inland areas tend to be more sparsely populated. Given Brunei’s unique geography, understanding how educational facilities are distributed across these areas can provide insights into the accessibility challenges faced by students in different regions. \\

In the study of GIS and spatial data analysis, Dr. Haziq Jamil (2024) highlights the importance of examining spatial patterns to address questions of access and resource allocation. This study builds on this approach by applying spatial autocorrelation methods to assess school clustering and distribution across Brunei. Although this research does not address school quality directly, it provides essential foundational data on spatial distribution that can support future investigations into educational outcomes and equity in Brunei. By evaluating the alignment between school locations and population density, this study offers valuable insights that can guide policymakers in optimizing educational infrastructure to ensure equitable access for all students. \\

SDG
Importance of educations etc
Topic Question:
	1. Are there any clusters?
	2. Where are the clusters located?
	3. Do the clusters match the population density?

Brunei mukim, kpg, pop info -  (quote haziqj paper). What does mukims mean in english.
Brunei is known for high wealth? Robust education? Quote Pisa stats. Free education. But no paper highlighting if spatial distribution of schools is efficient or not, whether it is equally for rich and poor. We want to see if this is a factor behind. This paper itself does not tackle the application but sets up the distribution needed for answering any future questions, such as quality/exam results/

\section{Literature Review}
\label{sec: lit rev}
 

\section{Description of Study Area and Data}
\label{sec: study area and data}
In this section, we provide an overview of the study area, Brunei Darussalam, and the data used in this study. A preliminary data analysis of the study variable is also presented.

\subsection{Description of Study Area}
\label{subsec: study area}
Brunei Darussalam, commonly known as Brunei, is located on the northern coast of the island of Borneo in Southeast Asia. With an area of approximately 5,765 square kilometers, Brunei is bordered by the South China Sea to the north and surrounded by the Malaysian state of Sarawak. The nation’s territory is divided into two non-contiguous areas:  The larger western section comprising Brunei-Muara, Tutong, and Belait districts; and the smaller eastern Temburong district. In the Northeast of the larger section lies Brunei's capital, Bandar Seri Begawan. 

The districts of Brunei are subdivided into 39 smaller administrative zones known as mukims, each embraces a number of kampung (villages). Brunei’s geography is characterized by a mix of urban centers, dense forests, and coastal lowlands. More than 70\% of the nation is covered with forests, with majority locating inland, southern parts of Belait and Tutong, as well as most of Temburong.

\begin{figure}[htbp]
\centering
\includegraphics[width=1\textwidth]{sch_all.pdf}
\caption{\label{fig:schools} Administrative zones of Brunei, showing the four main districts: Brunei-Muara, Tutong, Belait, and Temburong. Each point represents a school location. Grey lines indicate  kampong boundaries, with thicker lines marking mukim boundaries. Forested areas are shaded in green.}
\end{figure} 

% Population, Wealth , Education
According to the 2021 census, Brunei has a population of approximately 445,000. The majority of the population is concentrated along the coastline, particularly in Bandar Seri Begawan, which serves as the administrative, cultural, and economic center of the nation. Brunei is a high-income country, boasting the second-highest per capita income and Human Development Index (HDI) in Southeast Asia, as well as the highest per capita Gross National Income (GNI) among OECD countries from 2005 to 2020 [43].

Education in Brunei is both free (for citizens) and compulsory for children aged 5 to 16, leading to a high literacy rate across the population. Given the nation’s wealth and commitment to education, it would be interesting to leverage spatial analysis in finding patterns and understanding how schools are clustered and distributed across the country.

\subsection{Data Collection \& Processing}
\label{subsec: data collection}
The dataset comprises \( N = 252 \) schools in Brunei Darussalam, sourced from the Ministry of Education’s \textit{2018 Education Statistics Book}. The decision to use the 2018 dataset stems from the lack of detailed data in more recent publications, which only provide summary versions. Specifically, the 2018 dataset includes:

\begin{enumerate}
    \item A complete listing of all schools in Brunei by sector
    \item Categorization of pre-primary to sixth forms institutions from Ministry of Education (MOE Sector) into administrative clusters (Cluster 1–6)
    \item Student-teacher ratios and enrolment by sector and cluster,
\end{enumerate}

details which are not available in the summarised editions of the statistical book from recent years.

Since the \textit{2018 Education Statistics Book} is only available in PDF format, we converted it to a spreadsheet format using online converter. The data was then extracted, cleaned and reorganized in Excel before being imported into R using the \texttt{read\_csv()} function.

% on collecting locations?%
In order to retrieve the latitudes and longitude of the schools, the \texttt{osmdata\_sf()} function from the \textit{osmdata} package was initially used. This approach however proved insufficient, as some schools were missing, and others had abbreviated names. Consequently, only partial location data was obtained. To address this, \texttt{left\_join()} was used to merge the available locations with the school listing, and remaining coordinates was manually collected.

\begin{tcolorbox}[title=Corresponding Codes]
\begin{verbatim}
# 1. Query locations from OSM
bb <- getbb("brunei darussalam", 
            format_out = "polygon", 
            featuretype = "country")

q <- opq(bb[2]) %>% 
    add_osm_feature(key = "amenity", value = "school") %>% 
    osmdata_sf() %>% 
    trim_osmdata(bb[2])

sch_sf <- q$osm_polygons %>% 
        as_tibble() %>% 
        st_as_sf() %>%  
        select(osm_id, name) %>% 
        drop_na() %>% 
        st_centroid()

# 2. Join locations to matching schools
sch_df <- 
  left_join(sch_df, sch_sf_match, by = c("School" = "name")) %>% 
  select(School, Sector, Cluster, `Education Level`, geometry)

# 3a. Export joined dataset
sch_sf <- st_as_sf(sch_df)
sch_sf$longitude <- st_coordinates(sch_sf)[,1]
sch_sf$latitude <- st_coordinates(sch_sf)[,2]
sch_sf <- st_drop_geometry(sch_sf) %>% data.frame()
write.csv(sch_sf, "school listing v2.csv", row.names = FALSE)

# 3b. Manually add in remaining address using MS Excel
      New version saved as "school listing.xlsx"

# 4. Import and convert to sf 
sch_df <- read_excel("data/school listing.xlsx", 1)
sch_sf <- st_as_sf(sch_df, coords = c("longitude", "latitude"), 4326)
sch_sf <- st_join(sch_sf, kpg_sf, join = st_within)
\end{verbatim}
\end{tcolorbox}

\subsection{Preliminary Data Analysis}
\label{subsec: prelim data analysis}
Generally, it seems that schools in Brunei are located near the shoreline, particularly towards the South China Sea. 

\begin{figure}[htbp]
\centering
\includegraphics[width=1\textwidth]{sch_mkm.pdf}
\end{figure} 

Apart from Pusat Pembangunan Belia and Pusat Latihan Kesenian dan Pertukangan Tangan which serves as youth and community training centers, academic schools in Brunei are categorized into three main sectors: Ministry of Education (MOE), Ministry of Religious Affairs (MORA) and private institutions. The distribution of schools across these sectors includes 164 under MOE, 9 under MORA, and 77 private, comprising approximately 70\% public (MOE, MORA) and 30\% private. 

In the MOE sector, schools from pre-primary to sicth form are organized into Clusters 1 to 6. While the number of schools in each cluster is relatively balanced, Clusters 3 and 4 have notably higher class counts and student populations, followed by Clusters 1 and 2, with Clusters 5 and 6 having the lowest.

\begin{table}[htbp]
\centering
\setlength{\tabcolsep}{10pt}
\begin{tabular}{lrrr}
\hline
\textbf{Cluster} & \textbf{School Count} & \textbf{Class Count} & \textbf{Students Count} \\ 
\hline
Cluster 1        & 25                    & 453                  & 9,505                   \\
Cluster 2        & 26                    & 486                  & 9,606                   \\
Cluster 3        & 25                    & 566                  & 11,064                  \\
Cluster 4        & 27                    & 505                  & 10,648                  \\
Cluster 5        & 29                    & 379                  & 6,183                   \\
Cluster 6        & 21                    & 359                  & 6,884                   \\
\hline
\end{tabular}
\end{table}

\begin{figure}[htbp]
\centering
\includegraphics[width=1\textwidth]{sch_moe.pdf}
\end{figure} 

In regards to student teacher ratio, we concentrate on pre-primary through sixth-form schools, excluding vocational and higher education institutions due to their inconsistent structures and varying class arrangements. Among districts, Belait and Brunei-Muara have relatively higher values (about 10) compared to Temburong and Tutong (approximately 7.6). By sector, MOE and MORA school shares similar values, whereas private schools have nearly double the student teacher ratio, except in the Temburong district. 

\begin{table}[htbp]
\centering
\setlength{\tabcolsep}{10pt}
\begin{tabular}{lrrr}
\hline
\textbf{District}   & \textbf{Student}  & \textbf{Teacher}  & \textbf{Student-Teacher Ratio} \\ 
\hline
Belait              & 12,955            & 1,239             & 10.50 \\
Brunei Muara        & 68,188            & 6,892             & 9.89 \\
Temburong           & 1,893             & 248               & 7.63 \\
Tutong              & 9,029             & 1,180             & 7.65 \\ 
\hline
\end{tabular}
\caption{District-wise Student-Teacher Ratio}
\end{table}

\begin{table}[htbp]
\centering
\setlength{\tabcolsep}{15pt} % Adjusts horizontal padding
\begin{tabular}{lrrr}
\hline
\textbf{Sector} & \textbf{Student} & \textbf{Teacher} & \textbf{Student-Teacher Ratio} \\ 
\hline
MOE             & 53,890           & 6,574            & 8.20                           \\
MORA            & 5,483            & 670              & 8.18                           \\
Private         & 32,692           & 2,315            & 14.1                           \\
\hline
\end{tabular}
\caption{Sector-wise Student-Teacher Ratio}
\end{table}

\begin{figure}[h]
\centering
\includegraphics[width=1\textwidth]{student_teacher_ratio.pdf}
\end{figure} 

Due to the relatively low amount of schools in Brunei ($N = 252$), the spatial autocorrelation analysis in Section \ref{sec: methods} and Section \ref{sec: results} will consider all schools as whole.

\section{Methods}
\label{sec: methods}
This section provides detailed descriptions of the spatial autocorrelation methods used to analyse the hostpots and clusters of schools.

\subsection{Global spatial autocorrelation: Global Moran's I}
\label{subsec: gisa}
To examine whether schools in Brunei exhibit a clustered, dispersed, or random spatial pattern, we apply the Global Moran’s I test using the \texttt{global\_moran\_test()} function from \texttt{sfdep} package. This test is computed for each mukim (administrative division) in the study area, indexed by \( i, j = 1, 2, \ldots, N \). The Moran’s I test statistic is defined as follows: 

\begin{equation}
I = \frac{N}{\sum_{i=1}^N \sum_{i=1}^N w_{ij}} \frac{\sum_{i=1}^N \sum_{j=1}^N w_{ij} (x_i - \bar{x})(x_j - \bar{x})}{\sum_{i=1}^N (x_i - \bar{x})^2} \in [-1,1],
\end{equation}

where:
\begin{itemize}
    \item $x_i$ is the value of the study variable (count of schools) in mukim $i$,
    \item $\bar{x}$ is the mean number of schools per mukim,
    \item $w_{ij}$ is the spatial weight between mukims $i$ and $j$
\end{itemize}

For simplicity, rook contiguity neighbours is used for the spatial weights, as discussed in Section \ref{sec: lit rev}. This approach assigns $w_{ij} = 1$ if mukims $i$ and $j$ share one or more boundaries, and $w_{ij} = 0$ otherwise. 

Moran’s I values are standardized, with values close to $+1$ indicating positive spatial autocorrelation (i.e., clustering), where high or low values are near each other. Values close to $-1$ indicate negative spatial autocorrelation (i.e., dispersion), where neighboring values differ significantly. Values near $0$ suggest randomness, indicating an absence of spatial pattern. 

\begin{figure}[htbp]
\centering
\includegraphics[width=1\textwidth]{autocorrelation.pdf}
\end{figure} 

To determine the significance of the Moran’s I statistic, we employ the Central Limit Theorem to calculate p-values based on a Z-score, allowing us to test the following hypotheses:
\begin{itemize}
    \item \( H_0: I = 0 \) (no spatial autocorrelation),
    \item \( H_1: I \neq 0 \) (presence of spatial autocorrelation).
\end{itemize}

\begin{tcolorbox}[title=Corresponding Codes]
\begin{verbatim}
mkm_sch <- mkm_sf %>% 
            left_join(sch_mkm) %>% 
            select(mukim, count_of_schools)

mkm_sch$count_of_schools[is.na(mkm_sch$count_of_schools)] <- 0

nb <- st_contiguity(mkm_sch)
wt <- st_weights(nb)
global_moran_test(mkm_sch$count_of_schools, nb, wt)
\end{verbatim}
\end{tcolorbox}

\subsection{Local spatial autocorrelation (LISA): Getis-Ord $G_i^*$}
\label{subsec: lisa}
While a visual inspection suggests that certain \textit{kampungs} (villages) may have a higher concentration of schools, we aim to quantify this pattern. Whereas global spatial autocorrelation tests confirm whether clustering exists, we use the Getis-Ord $G_i^*$ statistic to identify the specific areas where schools are concentrated. This statistic is computed using the \texttt{hotspot\_gistar} function from \texttt{sfhotspot} package.

In our analysis, the study area is subdivided into $n$ rectangular grids, indexed by $i=1, 2, \ldots, n$. For each $i$-th grid cell, the $G_i^*$ statistic is calculated as:

\begin{equation}
G_i^* = \frac{\sum_j w_{ij} x_j}{\sum_j x_j}
\end{equation}

where:
\begin{itemize}
    \item $x_j$ is the value of the study variable (the count of schools) for grid cell $j$,
    \item $w_{ij}$ is the spatial weight between grid cells $i$ and $j$, with 1 indicating neighboring cells and 0 for non-neighboring cells.
\end{itemize}

Similar to global spatial autocorrelation in Section \ref{subsec: gisa}, the spatial weights used are based on rook contiguity negihbours. However, there is one slight modification: the spatial weight $w_{ii}$ are set to 1 rather than 0. This adjustment gives $G_i^*$ a more localized perspective, which is valuable for identifying clusters centered directly on a point of interest rather than merely in its surrounding areas.

A statistically significant high $G_i^*$ value indicates a “hotspot,” or a cluster of high values, whereas a low $G_i^*$ value suggests a “coldspot,” or a cluster of low values.

To only highlight significant clusters, the output was filtered to include only values with $G_i^* > 0$ and p-value $< 0.05$. The output dataset is then cropped to Brunei’s boundary using \texttt{st\_intersection} to refine the analysis. This method identifies school hotspots, areas where there are more schools than would be expected if they were distributed randomly. 

The corresponding codes are as follow:

\begin{tcolorbox}[title=Corresponding Codes]
\begin{verbatim}
sch_gi <- sch_sf %>% 
    st_transform("EPSG:27700") %>%  # sfhotspot rejects EPSG 4326 
    hotspot_gistar() %>%  
    filter(gistar > 0, pvalue < 0.05) %>% 
    st_intersection(st_union(dis_sf) %>% 
                    st_transform("EPSG:27700"))
\end{verbatim}
\end{tcolorbox}

\section{Results}
\label{sec: results}
The following section is split into 3 subsections corresponding to hypotheses as introduced in Section Introduction.

\subsection{Are there cluster?}
The Global Moran's I analysis yielded I value of \textbf{0.457}. The positive Moran’s I statistic suggests a positive autocorrelation in count of schools across mukims. Given the statistically significant results (low p-value of $\mathbf{4.542 \times 10^{-6}} < 0.001$), there is sufficient evidence to reject the null hypothesis $H_0$, which assumes no spatial autocorrelation in the distribution of schools. 

This finding supports the presence of moderate to strong clustering tendency, implying that mukims with a similar number of schools, whether high or low, are geographically close to each other. This results of the Global Moran’s I test aligns with our preliminary observation in Section \ref{subsec: prelim data analysis}, that there exists schools clusters. In the following subsection, we will further verify that these clusters are more concentrated near the coastal regions through local spatial autocorrelation.

\subsection{Locations of School Clusters}
\begin{figure}[htbp]
\centering
\includegraphics[width=1\textwidth]{sch_lisa.pdf}
\end{figure} 

As highlighted in Fig. x, the primary concentration of schools is located in central Bandar Seri Begawan, the capital district of Brunei. This is expected its status as the nation’s capital, urban and administrative center. Other notable clusters outside the capital are located in:
\begin{itemize}
    \item Temburong District: \textbf{Mukim Bangar}
    \item Tutong District: \textbf{Mukim Pekan Tutong}, \textbf{Mukim Telisai}
    \item Kuala Belait District: \textbf{Mukim Kuala Belait}, \textbf{Mukim Seria}
\end{itemize}

This result confirms that school clusters does indeed concentrate near the coastal regions. On top of this, it seems that schools may not be as abundant or accessible in the outskirts and outside of the capital district, Brunei-Muara. 

Another insight of Getis-Ord analysis is that it offers an advantage over the choropleth map in fig. x by pinpointing the exact areas of clustering within each mukim. For example, schools cluster in the northeastern areas of Mukim Telisai but are more concentrated toward the south in Mukim Bangar. This level of detail enables a more precise understanding of spatial clustering patterns. 

\subsection{Comparison to Distribution of Population}
\subsection{Locations of School Clusters}
\begin{figure}[htbp]
\centering
\includegraphics[width=1\textwidth]{bn_pop.pdf}
\end{figure} 

\subsection{Locations of School Clusters}
\begin{figure}[htbp]
\centering
\includegraphics[width=1\textwidth]{sch_kpg.pdf}
\end{figure} 

Although visually, schools and population hotspots seems to match. When comparing the top 10 kampongs with more schools and population in Table. x below, we see that only 3 kampongs: Kampong A, Kg. B, Kg. C are in both categories. This suggests that clusters of schools are not located in exactly the areas with most population, but perhaps nearby. 

\begin{table}[]
\begin{tabular}{@{}llc|lr@{}}
\toprule
                        & Kampong                                 & Schools count             & Kampong                                 & Population                     \\ \midrule
\multicolumn{1}{l|}{1}  & Kg. Kiarong                             & 6                         & Kg. Panchor Mentiri                     & 13,358                         \\
\multicolumn{1}{l|}{2}  & \cellcolor[HTML]{FFFFC7}Kg. Mata-Mata   & \cellcolor[HTML]{FFFFC7}6 & Kg. Tanah Jambu                         & 11,695                         \\
\multicolumn{1}{l|}{3}  & Kg. Parit Kianggeh                      & 6                         & \cellcolor[HTML]{FFFFC7}Kg. Panaga      & \cellcolor[HTML]{FFFFC7}10,301 \\
\multicolumn{1}{l|}{4}  & Pekan Kuala Belait                      & 6                         & Kg. Bukit Beruang                       & 9,835                          \\
\multicolumn{1}{l|}{5}  & Bukit Bendera                           & 6                         & Kg. Meragang                            & 9,190                          \\
\multicolumn{1}{l|}{6}  & \cellcolor[HTML]{FFFFC7}Kg. Panaga      & \cellcolor[HTML]{FFFFC7}6 & \cellcolor[HTML]{FFFFC7}Kg. Mata-Mata   & \cellcolor[HTML]{FFFFC7}7,159  \\
\multicolumn{1}{l|}{7}  & \cellcolor[HTML]{FFFFC7}Kg. Sungai Akar & \cellcolor[HTML]{FFFFC7}5 & Kg. Beribi                              & 6,490                          \\
\multicolumn{1}{l|}{8}  & Kg. Tungkadeh                           & 5                         & Kg. Kilanas                             & 6,357                          \\
\multicolumn{1}{l|}{9}  & Kg. Tungku                              & 5                         & \cellcolor[HTML]{FFFFC7}Kg. Sungai Akar & \cellcolor[HTML]{FFFFC7}6,129  \\
\multicolumn{1}{l|}{10} & Kg. Kiulap                              & 4                         & Kg. Mulaut                              & 5,981                          \\ \bottomrule
\end{tabular}
\end{table}

\begin{table}[]
\centering
\setlength{\tabcolsep}{10pt}
\begin{tabular}{@{}llc|lr@{}}
\toprule
                        & Mukim                                     & Schools count             & Mukim                                     & Population                     \\ \midrule
\multicolumn{1}{l|}{1}  & \cellcolor[HTML]{FFFFC7}Mukim Gadong B     & \cellcolor[HTML]{FFFFC7}24 & \cellcolor[HTML]{FFFFC7}Mukim Sengkurong  & \cellcolor[HTML]{FFFFC7}40,972 \\
\multicolumn{1}{l|}{2}  & \cellcolor[HTML]{FFFFC7}Mukim Berakas B    & \cellcolor[HTML]{FFFFC7}21 & Mukim Mentiri                             & 39,324                         \\
\multicolumn{1}{l|}{3}  & \cellcolor[HTML]{FFFFC7}Mukim Berakas A    & \cellcolor[HTML]{FFFFC7}21 & \cellcolor[HTML]{FFFFC7}Mukim Berakas B   & \cellcolor[HTML]{FFFFC7}39,284 \\
\multicolumn{1}{l|}{4}  & Mukim Kianggeh                             & 20                         & \cellcolor[HTML]{FFFFC7}Mukim Gadong B    & \cellcolor[HTML]{FFFFC7}38,067 \\
\multicolumn{1}{l|}{5}  & \cellcolor[HTML]{FFFFC7}Mukim Gadong A     & \cellcolor[HTML]{FFFFC7}14 & \cellcolor[HTML]{FFFFC7}Mukim Gadong A    & \cellcolor[HTML]{FFFFC7}35,424 \\
\multicolumn{1}{l|}{6}  & \cellcolor[HTML]{FFFFC7}Mukim Seria        & \cellcolor[HTML]{FFFFC7}12 & \cellcolor[HTML]{FFFFC7}Mukim Kuala Belait & \cellcolor[HTML]{FFFFC7}28,793 \\
\multicolumn{1}{l|}{7}  & Mukim Perkan Tutong                        & 12                         & \cellcolor[HTML]{FFFFC7}Mukim Berakas A   & \cellcolor[HTML]{FFFFC7}28,311 \\
\multicolumn{1}{l|}{8}  & \cellcolor[HTML]{FFFFC7}Mukim Kuala Belait & \cellcolor[HTML]{FFFFC7}11 & Mukim Kilanas                             & 24,981                         \\
\multicolumn{1}{l|}{9}  & \cellcolor[HTML]{FFFFC7}Mukim Sengkurong   & \cellcolor[HTML]{FFFFC7}10 & Mukim Serasa                              & 18,569                         \\
\multicolumn{1}{l|}{10} & Mukim Pangkalan Batu                       & 9                          & \cellcolor[HTML]{FFFFC7}Mukim Seria       & \cellcolor[HTML]{FFFFC7}18,313 \\ \bottomrule
\end{tabular}
\end{table}


\section{Conclusions}
By providing this pinpoints of school clusters, it is hoped that this can help in future housing/amenity/etc planning? \\
\label{sec: conclusions}
\section{Acknowledgements}
 The authors expresses gratitude OpenStreetMap contributors for the spatial data used in this analysis, and Dr Haziq Jamil, instructor of the module SM2302, for his invaluable insights
 which enriched this study.

 The brunei map and population dataset is sourced from propetyprice Github repo.  
\section{Reference}

\end{document}

